%**************************************************************************
%*
%*  Paper: ``INSTRUCTIONS FOR AUTHORS OF LATEX DOCUMENTS''
%*
%*  Publication: 2015 Winter Simulation Conference Author Kit
%*
%*  Filename: wsc15paper.tex
%*
%*  Date: January 31, 2001   Time:  9:45 PM
%*      BASE of current version: Feb 01, 2010 (primary WSC'10 LaTeX file)
%*
%*  Word Processing System: TeXnicCenter and MiKTeX
%*
%*
%*  All files need the following
\input{wsc15style.tex}     % download from author kit.  Style files for wsc formatting. Don't remove this line - required for generating the final paper!

\documentclass{wscpaperproc}
\usepackage{latexsym}
%\usepackage{caption}
\usepackage{graphicx}
\usepackage{mathptmx}

%
%****************************************************************************
% AUTHOR: You may want to use some of these packages. (Optional)
\usepackage{amsmath}
\usepackage{amsfonts}
\usepackage{amssymb}
\usepackage{amsbsy}
\usepackage{amsthm}
%****************************************************************************



%
%****************************************************************************
% AUTHOR: If you do not wish to use hyperlinks, then just comment
% out the hyperref usepackage commands below.

%% This version of the command is used if you use pdflatex. In this case you
%% cannot use ps or eps files for graphics, but pdf, jpeg, png etc are fine.

\usepackage[pdftex,colorlinks=true,urlcolor=blue,citecolor=black,anchorcolor=black,linkcolor=black]{hyperref}

%% The next versions of the hyperref command are used if you adopt the
%% outdated latex-dvips-ps2pdf route in generating your pdf file. In
%% this case you can use ps or eps files for graphics, but not pdf, jpeg, png etc.
%% However, the final pdf file should embed all fonts required which means that you have to use file
%% formats which can embed fonts. Please note that the final PDF file will not be generated on your computer!
%% If you are using WinEdt or PCTeX, then use the following. If you are using
%% Y&Y TeX then replace "dvips" with "dvipsone"

%%\usepackage[dvips,colorlinks=true,urlcolor=blue,citecolor=black,%
%% anchorcolor=black,linkcolor=black]{hyperref}
%****************************************************************************



		



%
%****************************************************************************
%*
%* AUTHOR: YOUR CALL!  Document-specific macros can come here.
%*
%****************************************************************************

% If you use theoremes
\newtheoremstyle{wsc}% hnamei
{3pt}% hSpace abovei
{3pt}% hSpace belowi
{}% hBody fonti
{}% hIndent amounti1
{\bf}% hTheorem head fontbf
{}% hPunctuation after theorem headi
{.5em}% hSpace after theorem headi2
{}% hTheorem head spec (can be left empty, meaning `normal')i

\theoremstyle{wsc}
\newtheorem{theorem}{Theorem}
\renewcommand{\thetheorem}{ \arabic{theorem}}
\newtheorem{corollary}[theorem]{Corollary}
\renewcommand{\thecorollary}{\arabic{corollary}}
\newtheorem{definition}{Definition}
\renewcommand{\thedefinition}{\arabic{definition}}


%#########################################################
%*
%*  The Document.
%*
\begin{document}

%***************************************************************************
% AUTHOR: AUTHOR NAMES GO HERE
% FORMAT AUTHORS NAMES Like: Author1, Author2 and Author3 (last names)
%
%		You need to change the author listing below!
%               Please list ALL authors using last name only, separate by a comma except
%               for the last author, separate with "and"
%
\WSCpagesetup{Li, and Chen}

% AUTHOR: Enter the title, all letters in upper case
\title{C\# PARADIGM FOR DISCRETE EVENT SIMULATION MODELING}

% AUTHOR: Enter the authors of the article, see end of the example document for further examples
\author{Haobin Li\\Yinxin Chen\\ [12pt]
	National University of Singapore.\\
	Department of Industrial and Systems Engineering\\
	1 Engineering Drive 2, 117576, SINGAPORE
}

\maketitle

\section*{ABSTRACT}

For complex industrial problems, the discrete event simulation (DES) is a powerful tool to evaluate the performances of system configuration and seek for the optimal solutions. And C\# is a fast evolving programming language that is designed to be simple, general-purpose, and object-oriented in developing a variety of applications that run on Microsoft .NET Framework. In this paper, we provide an alternative way to build DES model in C\# language, and demonstrate that compared with a commercial DES package the paradigm is compact and flexible, as it facilitates the integration with database, user-interface, or a simulation optimization infrastructure with advanced features, e.g., parallel evaluations and simulation budget control.

\section{INTRODUCTION}

Many industries have applied the discrete event simulation (DES) as an analytical tool to evaluate their system performance either in a deterministic and stochastic manner. In a long run, commercial software packages remain as the main approaches for such kinds of analysis, and AutoMod \shortcite{Muller2011}, FlexSim \shortcite{Nordgren2002}, and Arena \shortcite{Kelton1998} are the typical examples. Usually, the software packages provide a graphical interface for the users to build the model and is able to show fantastic animation for them to have a first-hand experience of what will be happening in the system.

% why we need a C# simulator?

Main advantage: easy for scientists and engineers with limited programing capabilities to build a flexible simulation model, for scenario analysis and optimization.

Some feature:

\begin{itemize}
	\item easy for UI with MVC
	\item easy for database with entity framework
	\item easy to integrate with optimizer in C\#
	\item easy for parallelized evaluation
	\item discontinuous evaluation, for sequential allocation of simulation budget and
	\item time-dilated evaluation
\end{itemize}

\section{BACKGROUND}

% a brief history of discrete event simulation

% how people build it?

% some most advantage techniques

% briefly discuss what is lacking? 

% what is provided in C# and .NET Framework, e.g., strong type and object-oriented, MVC, Entity Framework, parallel threads handling,  other scientific packages

\section{THE FRAMEWORK}

Talk about the generic components, including scheduled events, future event list, and clocks.
use class and sequence diagram to show how it works.

To build a simulation model, three categories of information should be added on the framework, i.e., scenario model, status model, and events.
refer to the example below.

\section{CASE STUDIES}

\subsection{M/M/1 Queue}

parallelization

\subsection{A Workshop Problem}

with time dilation

\subsection{Aircraft Spare Part Management}

UI and database

with optimization

\section{CONCLUSION}


% Please don't exchange the bibliographystyle style
\bibliographystyle{wsc}
% AUTHOR: Include your bib file here
\bibliography{ref}

\section*{AUTHOR BIOGRAPHIES}

\noindent {\bf HAOBIN LI} is Research Fellow at the Industrial and Systems Engineering Department of National University of Singapore. He received his B. Eng. (1st Class Hons) and Ph. D. degrees in 2009 and 2014 from the same department. His research interests include system modeling, multi-objective optimization and interactive information system. His email address is \email{i@li-haobin.net}.\\

\end{document}

